\begin{frame}{Building blocks}
    \begin{itemize}
        \item Building blocks for a symmetric block cipher: bitwise \texttt{XOR} with round key, Sbox, and permutation.
        \item It is easy in both software and hardware to implement bitwise \texttt{XOR} with a round key.
        \begin{itemize}
            \item In hardware, there is an \texttt{XOR} gate 
            \item almost every processor has a dedicated \texttt{XOR} instruction
        \end{itemize}
    \end{itemize}
\end{frame}

\begin{frame}{Implementing Sboxes}
    \begin{itemize}
        \item In software, a na\"ive way to implement Sbox is to use a lookup table. 
        \item The table is stored as an array in random access memory or flash memory.
        \item The storage space required for an Sbox SB$:\FF_2^{\omega_1}\to\FF_2^{\omega_2}$ is $\omega_2\times2^{\omega_1}$.
       \item For example, PRESENT has a $4-$bit Sbox and the storage required is $2^4\times4=64$ bits, or $8$ bytes.
    \end{itemize}
\end{frame}

\begin{frame}{AES -- T-tables}
    \begin{itemize}
        \item Implementation method that combines SubBytes, ShiftRows, and MixColumns for AES round function.
        \item SB: the AES Sbox.
        \item Let us denote the input of SubBytes by a matrix $S$
    \[
    S=
    \begin{pmatrix}
    s_{00} & s_{01} & s_{02} & s_{03}\\
    s_{10} & s_{11} & s_{12} & s_{13}\\
    s_{20} & s_{21} & s_{22} & s_{23}\\
    s_{30} & s_{31} & s_{32} & s_{33}
    \end{pmatrix}
    \vspace*{-0.2cm}
    \]
    \item $A, B, D$: outputs of SubBytes, ShiftRows, MixColumns
    \item $a_{ij}=\text{SB}(s_{ij}), 0\leq i,j<4$.
    \end{itemize}
\end{frame}

\begin{frame}{AES -- T-tables}
    \begin{itemize}
    \item SB: AES Sbox.
    \item $S$: input of SubBytes
    \item $A, B, D$: outputs of SubBytes, ShiftRows, MixColumns
    \item $a_{ij}=\text{SB}(s_{ij}), 0\leq i,j<4$.
    \item For $j=0,1,2,3$
    \end{itemize}
\[
\begin{pmatrix}
    b_{0j}\\
    b_{1j}\\
    b_{2j}\\
    b_{3j}
    \end{pmatrix}
    =
    \begin{pmatrix}
    a_{0j}\\
    a_{1(j+1\mo 4)}\\
    a_{2(j+2\mo 4)}\\
    a_{3(j+3\mo 4)}
    \end{pmatrix},\quad
    \begin{pmatrix}
    d_{0j}\\
    d_{1j}\\
    d_{2j}\\
    d_{3j}
    \end{pmatrix}
    =
    \begin{pmatrix}
    \texttt{02} & \texttt{03} & \texttt{01} & \texttt{01} \\
    \texttt{01} & \texttt{02} & \texttt{03} & \texttt{01} \\
    \texttt{01} & \texttt{01} & \texttt{02} & \texttt{03} \\
    \texttt{03} & \texttt{01} & \texttt{01} & \texttt{02}
    \end{pmatrix}
    \begin{pmatrix}
    b_{0j}\\
    b_{1j}\\
    b_{2j}\\
    b_{3j}
    \end{pmatrix},
\]
\end{frame}

\begin{frame}{AES -- T-tables}
$D$: output of MixColumns
    \begin{eqnarray*}
    \begin{pmatrix}
    d_{0j}\\
    d_{1j}\\
    d_{2j}\\
    d_{3j}
    \end{pmatrix}
    &=&
    \begin{pmatrix}
    \texttt{02} & \texttt{03} & \texttt{01} & \texttt{01} \\
    \texttt{01} & \texttt{02} & \texttt{03} & \texttt{01} \\
    \texttt{01} & \texttt{01} & \texttt{02} & \texttt{03} \\
    \texttt{03} & \texttt{01} & \texttt{01} & \texttt{02}
    \end{pmatrix}
    \begin{pmatrix}
    \text{SB}(s_{0j})\\
    \text{SB}(s_{1(j+1\mo 4)})\\
    \text{SB}(s_{2(j+2\mo 4)})\\
    \text{SB}(s_{3(j+3\mo 4)})
    \end{pmatrix}\\
    &=&
    \begin{pmatrix}
    \texttt{02}\\
    \texttt{01}\\
    \texttt{01}\\
    \texttt{03}
    \end{pmatrix}\text{SB}(s_{0j})
    \oplus
    \begin{pmatrix}
    \texttt{03}\\
    \texttt{02}\\
    \texttt{01}\\
    \texttt{01}
    \end{pmatrix}\text{SB}(s_{1(j+1\mo 4)})\\
    &&\oplus
    \begin{pmatrix}
    \texttt{01}\\
    \texttt{03}\\
    \texttt{02}\\
    \texttt{01}
    \end{pmatrix}\text{SB}(s_{2(j+2\mo 4)})
    \oplus
    \begin{pmatrix}
    \texttt{01}\\
    \texttt{01}\\
    \texttt{03}\\
    \texttt{02}
    \end{pmatrix}\text{SB}(s_{3(j+3\mo 4)}),
    \end{eqnarray*}
\end{frame}

\begin{frame}{AES -- T-tables}
For $a\in\FF_2^8$, define
\begin{equation*}
    \begin{array}{ll}
    T_0(a):=\begin{pmatrix}
    \texttt{02}\\
    \texttt{01}\\
    \texttt{01}\\
    \texttt{03}
    \end{pmatrix}\text{SB}(a), & 
    T_1(a):=\begin{pmatrix}
    \texttt{03}\\
    \texttt{02}\\
    \texttt{01}\\
    \texttt{01}
    \end{pmatrix}\text{SB}(a),\\ 
      T_2(a):=\begin{pmatrix}
    \texttt{01}\\
    \texttt{03}\\
    \texttt{02}\\
    \texttt{01}
    \end{pmatrix}\text{SB}(a), & 
    T_3(a):=\begin{pmatrix}
    \texttt{01}\\
    \texttt{01}\\
    \texttt{03}\\
    \texttt{02}
    \end{pmatrix}\text{SB}(a).
    \end{array}
\end{equation*}
Then
\[
\begin{pmatrix}
    d_{0j}\\
    d_{1j}\\
    d_{2j}\\
    d_{3j}
    \end{pmatrix}
    =T_0(s_{0j})\oplus T_1(s_{1(j+1\mo 4)})
    \oplus T_2(s_{2(j+2\mo 4)})T_3(s_{3(j+3\mo 4)}),
\]
\end{frame}

\begin{frame}{AES -- T-tables}
    \begin{itemize}
        \item The four tables $T_0, T_1, T_2, T_3$ of size $8\times32$ can be used to implement SubBytes, ShiftRows, and MixColumns
       \item Those four tables are called \textit{T-tables} for AES.
       \item To store the T-tables we need processors with a word length of $32$ or above.
       \item They cannot be used for the last round of AES as there is no Mixcolumns operation.
    \end{itemize}
\end{frame}


\begin{frame}{PRESENT Sbox -- lookup table}
PRESENT Sbox:
\begin{table}
\centering
\ttfamily
\begin{tabular}{cccccccccccccccc}\hline
 0 & 1 & 2 & 3 & 4 & 5 & 6 & 7 & 8 & 9 & A & B & C & D & E & F \\\hline
 C & 5 & 6 & B & 9 & 0 & A & D & 3 & E & F & 8 & 4 & 7 & 1 & 2\\\hline
\end{tabular}
\end{table}
\begin{algorithm}[H]
	\textbf{integer array} [1..16] Sbox = \{\texttt{C}, \texttt{5}, \texttt{6}, \texttt{B}, \texttt{9}, \texttt{0}, \texttt{A}, \texttt{D}, \texttt{3}, \texttt{E}, \texttt{F}, \texttt{8}, \texttt{4}, \texttt{7}, \texttt{1}, \texttt{2}\}\\
 s = Sbox[s] \tcp{Table lookup}
 \caption{A lookup table implementation of PRESENT Sbox in pseudocode.}
\end{algorithm}
\begin{itemize}
        \item As current computer architectures normally use word sizes of at least one byte (generally multiple bytes), it is not efficient to implement Sbox nibble-wise.
To optimize the execution time, we can merge two PRESENT Sbox lookup tables
    \end{itemize}
\end{frame}

\begin{frame}{PRESENT Sbox -- lookup table}
    \begin{itemize}
        \item To optimize the execution time, we can merge two PRESENT Sbox lookup tables
    \end{itemize}
\begin{algorithm}[H]
	\textbf{integer array} [1..16] Sbox =\{\texttt{C}, \texttt{5}, \texttt{6}, \texttt{B}, \texttt{9}, \texttt{0}, \texttt{A}, \texttt{D}, \texttt{3}, \texttt{E}, \texttt{F}, \texttt{8}, \texttt{4}, \texttt{7}, \texttt{1}, \texttt{2}\}\\
 \textbf{integer} big\_s = Sbox[s \& 0x0F] \tcp{Lower nibble; $\&$: bitwise \texttt{AND}}
 big\_s = big\_s $\vee$ (Sbox[(s$\gg$4) \& 0x0F] $\ll$4) \tcp{Upper nibble; $\vee$: bitwise \texttt{OR}}
 s = big\_s \tcp{State update}
 \caption{A more efficient lookup table implementation of PRESENT Sbox in pseudocode.}
\end{algorithm}
\end{frame}

\begin{frame}{PRESENT Sbox -- lookup table}
    \begin{itemize}
        \item To avoid the bit shifts and boolean operations, it is better to combine two $4\times4$ Sbox tables into one bigger $8\times8$ table
    \end{itemize}
\begin{algorithm}[H]
	\textbf{integer array} [1..256] Sbox = \{\texttt{CC}, \texttt{C5}, $\dots$, \texttt{C1}, \texttt{C2}, \texttt{5C}, \texttt{55}, $\dots$, \texttt{51}, \texttt{52}, $\dots$ \texttt{2C}, \texttt{25}, $\dots$, \texttt{21}, \texttt{22}\}\\
 s = Sbox[s] \tcp{Table lookup of two nibbles in parallel}
\caption{A lookup table implementation combining two PRESENT Sboxes in parallel in pseudocode.}
\end{algorithm}
\end{frame}

\begin{frame}{Implementing permutations}
    \begin{itemize}
        \item The efficiency of the implementation is highly dependent on the design of the permutation.
        \item For AES ShiftRows, the bytes are permuted, making it easier to implement.
        \item For PRESENT pLayer, the bit level permutations are ``free'' in hardware as we just need to reorder the wires, no new gates are required.
        \item However, in software, extracting each bit and putting it in the right position is time-consuming.
    \end{itemize}
\end{frame}

\begin{frame}{PRESENT -- combine sBoxLayer and pLayer}
    \begin{itemize}
        \item Construct sixteen $4\times64$ lookup tables, TB1, TB2, $\dots$, TB16.
       \item The input of TBi is given by the $i$th nibble of the cipher state at the input of sBoxLayer.
       \item The outputs are $64-$bit values with mostly $0$s except for $4$ bits that correspond to the input nibble.
    \end{itemize}
\begin{alertblock}{Remark}
For PRESENT specification, we consider the $0$th bit of a value as the rightmost bit in its binary representation.
For example, the $0$th bit of $3=011_2$ is $1$, the $1$st bit is $1$ and the $2$nd bit is $0$.
\end{alertblock}
\end{frame}

\begin{frame}{PRESENT -- combine sBoxLayer and pLayer}
\begin{itemize}
    \item TB1: input is the first nibble of the cipher state at the input of sBoxLayer.
\item The Sbox output corresponding to this nibble should go to bits $0,16,32$ and $48$ of the output of pLayer.
\item Each entry of TB1 is a $64-$bit value with bits in positions $0,16,32$ and $48$ given by the Sbox output, and the other bits are all $0$.
\end{itemize}
\begin{example}
If the input is \texttt{A}, the Sbox output should be $\texttt{F}=1111_2$ and
\[
\text{TB1}[\texttt{A}]=0\dots010\dots010\dots010\dots1,
\]
where the $0$th, $16$th, $32$nd and $48$th bits are $1$.
\end{example}
\begin{table}
\centering
\ttfamily
\begin{tabular}{cccccccccccccccc}\hline
 0 & 1 & 2 & 3 & 4 & 5 & 6 & 7 & 8 & 9 & A & B & C & D & E & F \\\hline
 C & 5 & 6 & B & 9 & 0 & A & D & 3 & E & F & 8 & 4 & 7 & 1 & 2\\\hline
\end{tabular}
\caption{PRESENT Sbox}
\end{table}
\end{frame}

\begin{frame}{PRESENT -- combine sBoxLayer and pLayer}
\begin{itemize}
    \item TB1: input is the first nibble of the cipher state at the input of sBoxLayer.
\item The Sbox output corresponding to this nibble should go to bits $0,16,32$ and $48$ of the output of pLayer.
\item Each entry of TB1 is a $64-$bit value with bits in positions $0,16,32$ and $48$ given by the Sbox output, and the other bits are all $0$.
\end{itemize}
\begin{example}
PRESENT Sbox output for input \texttt{B} is $1000_2$, and
\[
\text{TB1}[\texttt{B}]=0\dots010\dots0,
\]
where the $48$th bit is $1$.
\end{example}
\end{frame}

\begin{frame}{PRESENT -- combine sBoxLayer and pLayer}
\begin{table}
\centering
\begin{tabular}{cccccccccccccccc}\hline
0 & 1 & 2 & 3 & 4 & 5 & 6 & 7 & 8 & 9 & 10 & 11 & 12 & 13 & 14 & 15 \\
0 & 16 & 32 & 48 & 1 & 17 & 33 & 49 & 2 & 18 & 34 & 50 & 3 & 19 & 35 & 51 \\\hline
\end{tabular}
\end{table}
\begin{example}
TB2 takes the second nibble of the cipher state as input.
The output bits should be positioned at $1,17,33$ and $49$.
Thus
\[
\text{TB2}[\texttt{B}]=0\dots010\dots0,
\]
where only the $49$th bit is $1$.
\end{example}
\end{frame}

\begin{frame}{PRESENT -- combine sBoxLayer and pLayer}
    \begin{itemize}
        \item A $4\times64$ table takes $64\times2^4$ bits and those sixteen tables take $16384$ bits of memory.
        \item Compared to one Sbox table, which is $64$ bits, this is much bigger, but these tables also implement pLayer of PRESENT.
       \item The speed can be further improved by merging two Sbox computations and constructing eight $8\times64$ lookup tables.
       \begin{itemize}
           \item The memory consumption will be the same, $16384$ bits.
           \item But the speed will be much faster.
       \end{itemize}
    \end{itemize}
\end{frame}