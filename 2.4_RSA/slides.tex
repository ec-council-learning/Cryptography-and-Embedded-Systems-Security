\section{Introduction}
\begin{frame}{\VideoName}
    \tableofcontents[currentsection]
\end{frame}

\begin{frame}{Key exchange}
    \begin{itemize}
        \item For symmetric key ciphers, a prior communication of the master key (\textit{key exchange}) is required before any ciphertext is transmitted.
        \item With only a symmetric key cipher, the key exchange may be difficult to achieve due to, e.g. far distance, and too many parties involved.
        \item In practice, this is where asymmetric key cryptosystem comes into use.
        \item For example, Alice would like to communicate with Bob using AES.
        \begin{itemize}
            \item To exchange the master key, $k$, for AES, she will encrypt $k$ by a public key cryptosystem using Bob's public key $e$, $c=E_e(k)$.
            \item The resulting ciphertext $c$ will be sent to Bob, and Bob can decrypt it with his secret private key $d$, $k=D_d(c)$.
            \item Then Alice and Bob can communicate with key $k$ using AES.
        \end{itemize}
    \end{itemize}
\end{frame}


\begin{frame}{Security of public key cryptosystem}
    \begin{itemize}
        \item Clearly, we require that it is computationally infeasible to find the private key $d$ given the public key $e$.
         \item In practice, this is guaranteed by some intractable problem
         \begin{itemize}
             \item A problem is intractable if there does not exist an efficient algorithm to solve it.
         \end{itemize}
        \item However, the cipher might not be secure in the future.
        \begin{itemize}
            \item For example, if a quantum computer with enough bits is manufactured, it can break many public key cryptosystems
        \end{itemize}
       \item A public key cipher is not perfectly secure 
       \begin{itemize}
           \item perfectly secure: if an attacker only has a collection ciphertexts, they cannot obtain any information about the plaintext no matter how much computing power they have
           \item however, the attacker can brute force the key
       \end{itemize}
    \end{itemize}
\end{frame}

\begin{frame}{Greatest common divisor}
\begin{definition}
Take $m,n\in\ZZ$, $m\neq 0$ or $n\neq 0$, the \textit{greatest common divisor} of $m$ and $n$, denoted $\gcd (m,n)$, is given by $d\in\ZZ$ such that
    \begin{itemize}
        \item $d>0$,
        \item $d|m$, $d|n$, and
        \item if $c|m$ and $c|n$, then $c|d$.
\end{itemize}
\end{definition}
\begin{example}
\begin{itemize}
    \item All positive divisors of $4$ and $6$ are $1,2,4$ and $1,2,3,6$ respectively.
So $\gcd(4,6)=2$.
\item All the positive divisors of $2$ are $1$ and $2$.
All the positive divisors of $3$ are $1$ and $3$.
So $\gcd(2,3)=1$.
\end{itemize}
\end{example}
\end{frame}

\begin{frame}{Prime numbers}
    \begin{definition}
\begin{itemize}
    \item For $m,n\in\ZZ$ such that $m\neq0$ or $n\neq0$, $m$ and $n$ are said to be \textit{relatively prime}/\textit{coprime} if $\gcd(m,n)=1$.
    \item Given $p\in\ZZ$, $p>1$.
    $p$ is said to be \textit{prime} (or a \textit{prime number}) if for any $m\in\ZZ$, either $m$ is a multiple of $p$ (i.e. $p|m$) or $m$ and $p$ are coprime (i.e. $\gcd(p,m)=1$).
\end{itemize}
\end{definition}
\begin{example}
\begin{itemize}
    \item $4$ and $9$ are relatively prime
    \item $8$ and $6$ are not relatively prime
    \item $2,3,5,7$ are prime numbers
    \item $6,9,21$ are not prime numbers
\end{itemize}
\end{example}
\end{frame}

\begin{frame}{Addition and multiplication in $\ZZ_n$}
\begin{itemize}
    \item Take $n>1$ an integer
    \item Take $\ZZ_n=\{0,1,\dots,n-1\}$, the set of remainders when dividing by $n$
    \item We define addition and multiplication in $\ZZ_n$ to be addition and multiplication \textit{modulo $n$}, i.e. computing the remainder of the sum/product dividing by $n$
\end{itemize}
\begin{example}
    \begin{itemize}
        \item Let $n=7$. $\ZZ_7=\{0,1,2,3,4,5,6\}$
        \[
        2+6\mo7=8\mo7=1,\quad 2\times6\mo7=12\mo7=5
        \]
        \item Let $n=5$, $\ZZ_5=\{0,1,2,3,4\}$
        \[
        2+4\mo5=6\mo5=1,\quad 2\times4\mo5=8\mo5=3
        \]
    \end{itemize}
\end{example}
\end{frame}

\begin{frame}{$\ZZ_n^*$}
\begin{definition}
\[
\ZZ_n^*:=\Set{a\ |\ a\in\ZZ_n,\ \gcd(a,n)=1}.
\]
Let $\varphi(n)$ denote the cardinality of $\ZZ_n^*$, i.e. number of elements in $\ZZ_n^*$
\[
\varphi(n)=\lvert \ZZ_n^*\rvert.
\]
$\varphi$ is called the \textit{Euler's totient function}.
\end{definition}
\begin{example}
    \begin{itemize}
        \item Let $n=3$, $\ZZ_3^*=\Set{1,2}$, $\varphi(3)=2$.
        \item Let $n=4$, $\ZZ_4^*=\Set{1,3}$, $\varphi(4)=2$.
        \item Let $n=p$ be a prime number, $\ZZ_p^*=\ZZ_p-\Set{0}=\Set{1,2,\dots,p-1}$, $\varphi(p)=p-1$.
    \end{itemize}
    \end{example}
\end{frame}

\section{RSA}
\begin{frame}{\VideoName}
    \tableofcontents[currentsection]
\end{frame}

\begin{frame}{RSA}
    \begin{itemize}
        \item Published in 1977
        \item Named after its inventors Ron Rivest, Adi Shamir, and Leonard Adleman.
      \item RSA is the first public key cryptosystem, and still in use today.
       % \item The security relies on the difficulty of finding the factorization of a composite positive integer.
    \end{itemize}
\end{frame}

\begin{frame}{Definition}
\begin{definition}[RSA]
Let $n=pq$, where $p,q$ are distinct prime numbers.
Let $\mathcal{P}=\mathcal{C}=\ZZ_n$, $\mathcal{K}=\ZZ_{\varphi(n)}^*-\Set{1}$.
For any $e\in \mathcal{K}$, define encryption
\[
    E_e:\ZZ_n\to\ZZ_n,\quad m\mapsto m^e\mo n,
\]
and the corresponding decryption
\[
    D_d:\ZZ_n\to\ZZ_n,\quad c\mapsto c^d\mo n,
\]
where $de\mo\varphi(n)=1$.
The cryptosystem $(\mathcal{P},\mathcal{C},\mathcal{K},\mathcal{E},\mathcal{D})$, where $\mathcal{E}=\Set{E_e:e\in\mathcal{K}}$, $\mathcal{D}=\Set{D_d:d\in\mathcal{K}}$, is called \textit{RSA}.
\end{definition}
\begin{itemize}
    \item $\varphi(n)=(p-1)(q-1)$
    \item Public key: $n,e$, RSA modulus, encryption exponent
    \item Private key: $d$, decryption exponent
\end{itemize}
\end{frame}

\begin{frame}{Key generation}
    \begin{itemize}
        \item Generate randomly and independently two large prime numbers $p$ and $q$.
         \item Compute $n=pq$.
         \begin{itemize}
             \item Normally $p$ and $q$ are supposed to have equal lengths.
             \item For example, take $p$ and $q$ to be 512-bit primes, and $n$ will be a 1024-bit modulus.
         \end{itemize}
          \item Choose $e\in\ZZ_{\varphi(n)}^*$
          \begin{itemize}
              \item Note that $e$ is odd since $\varphi(n)$ is even
              \item In practice, $e$ is chosen to be small to make the encryption efficient.
              \item However, $e$ cannot be too small. It has been shown that only the $n/4$ least significant bits of $d$ suffice to recover $d$ in the case of a small $e$
        \end{itemize}
      \item Compute $d$ using \textit{extended Euclidean algorithm}
     \begin{itemize}
         \item $d$ cannot be too small, it was proven that if $d<n^{0.292}$, then RSA can be broken
     \end{itemize}
    \end{itemize}
\end{frame}

\begin{frame}{RSA -- Example}
\begin{example}
\begin{itemize}
    \item As a toy example, suppose Bob would like to generate his private and public keys for RSA.
    \item Bob randomly generates $p=3$ and $q=5$.
    \item Then he computes $n=15$ and 
\[
\varphi(n)=2\times4=8.
\]
\[
\ZZ_{\varphi(n)}^*=\Set{1,3,5,7}
\]
\item From $\ZZ_{8}^*$, Bob chooses $e=3$.
\item Then by the extended Euclidean algorithm, he computes $d=3$.
We can check that
\[
de\mo8=3\times3\mo 8=1.
\]
\end{itemize}
\end{example}
\end{frame}

\begin{frame}{RSA -- Example}
\begin{example}
\[
p=3,\quad q=5,\quad n=15,\quad \varphi(n)=2\times4=8,\quad e=3,\quad d=3.
\]
\begin{itemize}
    \item Suppose Alice would like to send plaintext $m=2$ to Bob, using Bob's public key $n=15,e=3$.
    \item Alice computes 
\[
c=m^e\mo n=2^3\mo15=8.
\]
\item After receiving the ciphertext $c$ from Alice, Bob computes the plaintext using his private key 
\[
m=c^d\mo n=8^3\mo15=512\mo15=2.
\]
\end{itemize}
\end{example}
\end{frame}

\begin{frame}{Security of RSA}
    \begin{itemize}
        \item If $p$ or $q$ is known to the attacker
        \begin{itemize}
            \item can factorize $n$ and compute $\varphi(n)$
            \item with $e$, $d$ can be computed using the extended Euclidean algorithm 
        \end{itemize}
        \item All $p,q,\varphi(n)$ should be kept secret
        \item Of course, if the attacker can factorize $n$ with an efficient algorithm, then RSA is broken.
        \begin{itemize}
            \item Up to now, the best-known algorithm for integer factorization has been used to factorize RSA modulus of bit length $768$
            \item In practice, the most commonly used RSA modulus $n$ is 1024, 2048, or 4096 bit.
            \item On the other hand, there is no proof that factorizing an integer $n$ is infeasible.
        \end{itemize}
         \item It is not proven that RSA is secure if factoring is computationally infeasible -- there might be other ways to attack RSA.
    \end{itemize}
\end{frame}

\section{RSA Signatures}
\begin{frame}{\VideoName}
    \tableofcontents[currentsection]
\end{frame}

\begin{frame}{Digital signatures}
    \begin{itemize}
        \item Digital signatures provide means for an entity to bind its identity to a message. 
        \item This normally means that the sender uses their private key to sign the (hashed) message. 
        \item Whoever has access to the public key can then verify the origin of the message.
        \item For example, the message can be electronic contracts or electronic bank transactions.
        \item Suppose Alice signs a message $m$ with a private key $d$ and generates signature $s$.
        \item Bob receives the message and the signature, he can then verify $s$ with public key $e$ and a \textit{verification algorithm}.
        \item Given $m$ and $s$, the verification algorithm returns true to indicate a valid signature and false otherwise.
    \end{itemize}
\end{frame}

\begin{frame}{RSA signatures}
    \begin{itemize}
        \item To use RSA for digital signature, we again let $p$ and $q$ be two distinct primes.
        \item $n=pq$, choose $e\in\ZZ_{\varphi(n)}^*$, compute $d$ such that $de\mo\varphi(n)=1$
        \item The public key consists of $e$ and $n$.
        \item $d$ is the private key.
        \item $p$, $q$ and $\varphi(n)$ should be kept secret.
    \end{itemize}
\end{frame}

\begin{frame}{RSA signatures}
To sign a message $m$, Alice computes the signature
\[
s=m^d\mo n.
\]
Then Alice sends both $m$ and $s$ to Bob.
To verify the signature, Bob computes
\[
s^e\mo n.
\]
If $s\equiv m\mo n$, then the verification algorithm outputs true, and false otherwise.
\begin{itemize}
    \item Up to now, the only method known to compute $s$ from $m\mo n$ is using $d$, so if the verification algorithm outputs true, Bob can conclude that Alice is the owner of $d$.
\end{itemize}
\end{frame}

\begin{frame}{RSA signatures -- Example}
\begin{example}
\begin{itemize}
    \item Alice chooses $p=5$ and $q=7$.
    \item Then $n=35$ and $\varphi(n)=24$.
    \item Suppose Alice chooses $e=5$, which is coprime to $24$.
    \item  By the extended Euclidean algorithm Alice computes $d=5$
     \item To sign message $m=10$, Alice computes
    \[
    s=m^d\mo n=?
    \]
    \item Alice sends both the message and signature to Bob.
    \item Bob verifies the signature
    \[
    s^e\mo n= 5^5\mo 35 = 10 = m.
    \]
\end{itemize}
\end{example}
\end{frame}