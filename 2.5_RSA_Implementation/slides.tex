\section{Implementation of Modular Exponentiation}
\begin{frame}{\VideoName}
    \tableofcontents[currentsection]
\end{frame}

\begin{frame}{Modular exponentiation}
    \begin{itemize}
        \item To implement RSA or RSA signatures, we need to compute $a^d\mo n$ for some integer $a\in \ZZ_n$, \item $n=pq$ is a product of two distinct primes and $d\in\ZZ_{\varphi(n)}^*$.
        \item We can compute $d-1$ modular multiplications.
         \begin{itemize}
             \item inefficient for large $d$
             \item impossible for practical values of $d$ -- bit length more than $1000$
         \end{itemize}
         \item Square and multiply algorithm
    \end{itemize}
\end{frame}

\begin{frame}{Square and multiply algorithm}
    \begin{itemize}
        \item Let $n\geq2$ be an integer, $d\in\ZZ_{\varphi(n)}$, $a\in\ZZ_n$
        \item Binary representation of $d=d_{\ell_d-1}\dots d_2d_1d_0$, where $d_i=0,1$ and
\[
d=\sum_{i=0}^{\ell_d-1}d_i2^i.
\]
\item We have
\[
a^d=a^{\sum_{i=0}^{\ell_d-1}d_i2^i}=\prod_{i=0}^{\ell_d-1}(a^{2^i})^{d_i}=\prod_{0\leq i<\ell_d,d_i=1}a^{2^i}.
\]
\item Thus, to compute $a^d\mo n$, we can 
\begin{itemize}
    \item First compute $a^{2^i}$ for $0\leq i<\ell_d$
    \item Then $a^d$ is a product of $a^{2^i}$ for which $d_i=1$
\end{itemize}
\item Compared to $d-1$ modular multiplications, this requires at most $2\log_2 d$ multiplications
    \end{itemize}
\end{frame}

\begin{frame}{Square and multiply algorithm}
    \begin{algorithm}[H]
\KwIn{$n,\ a,\ d$\tcp{$n\in\ZZ, n\geq2$; $a\in\ZZ_n$; $d\in\ZZ_{\varphi(n)}$ has bit length $\ell_d$}}
\KwOut{$a^d\mo n$}
result $= 1$, $t = a$\\
	\For{$i=0$, $i<\ell_d$, $i++$}
 	{
 	\tcp{$i$th bit of $d$ is $1$}
  		\If{$d_i=1$}{
            \tcp{mutiply by $a^{2^i}$}
  		$\text{result} = \text{result}*t\mo n$\tcp{$
a^d=\displaystyle\prod_{0\leq i<\ell_d,d_i=1}a^{2^i}$}
  		}
  		\tcp{$t=a^{2^{i+1}}$}
  		$t=t*t\mo n$\\
  	}
  	\Return result\\
	\caption{Right-to-left square and multiply algorithm for computing modular exponentiation}
\end{algorithm}
\end{frame}


\begin{frame}{Right-to-left square and multiply algorithm}
\begin{columns}[T] % align columns
\begin{column}{.4\textwidth}
{
\setlength{\interspacetitleruled}{0pt}%
\setlength{\algotitleheightrule}{0pt}%
\begin{algorithm}[H]
result $= 1$, $t = a$\\
	\For{$i=0$, $i<\ell_d$, $i++$}
 	{
 	\tcp{$i$th bit of $d$ is $1$}
  		\If{$d_i=1$}{
            \tcp{mutiply by $a^{2^i}$}
  		$\text{result} = \text{result}*t\mo n$\\
  		}
  		\tcp{$t=a^{2^{i+1}}$}
  		$t=t*t\mo n$\\
  	}
  	\Return result
\end{algorithm}
}
\end{column}%
\hfill%
\begin{column}{.6\textwidth}
\begin{example}
    Let $n=15,\ d=3=11_2,\  a=2$.
    Then
    \[
    a^d\mo n=2^3\mo15=8\mo15=8
    \]
 \begin{center}
        \begin{tabular}{c|c|c|c}
     $i$ & $d_i$  &  $t$  & result\\\hline
     $0$ & $1$    &  $4$  &  $2$ \\
     $1$ & $1$    &  $1$  &  $8$
    \end{tabular}
    \end{center}
\end{example}
\end{column}%
\end{columns}
\end{frame}

\begin{frame}{Right-to-left square and multiply algorithm}
\begin{columns}[T] % align columns
\begin{column}{.4\textwidth}
{
\setlength{\interspacetitleruled}{0pt}%
\setlength{\algotitleheightrule}{0pt}%
\begin{algorithm}[H]
result $= 1$, $t = a$\\
	\For{$i=0$, $i<\ell_d$, $i++$}
 	{
 	\tcp{$i$th bit of $d$ is $1$}
  		\If{$d_i=1$}{
            \tcp{mutiply by $a^{2^i}$}
  		$\text{result} = \text{result}*t\mo n$\\
  		}
  		\tcp{$t=a^{2^{i+1}}$}
  		$t=t*t\mo n$\\
  	}
  	\Return result
\end{algorithm}
}
\end{column}%
\hfill%
\begin{column}{.6\textwidth}
\begin{example}
    Let $n=23,\ d=4=100_2,\  a=5$.
    Then 
    \[
    a^d\mo n=5^4\mo23=625\mo23=4
    \]
 \begin{center}
        \begin{tabular}{c|c|c|c}
     $i$ & $d_i$  &  $t$  & result\\\hline
     $0$ & $0$    &  $2$  &  $1$ \\
     $1$ & $0$    &  $4$  &  $1$ \\
     $2$ & $1$    &  $16$  &  $4$ \\
    \end{tabular}
    \end{center}
\end{example}
\end{column}%
\end{columns}
\end{frame}

\begin{frame}{Left-to-right square and multiply algorithm}
    \begin{algorithm}[H]
\KwIn{$n,\ a,\ d$\tcp{$n\in\ZZ, n\geq2$; $a\in\ZZ_n$; $d\in\ZZ_{\varphi(n)}$}}
\KwOut{$a^d\mo n$}
$t = 1$\\
	\For{$i=\ell_d-1$, $i\geq0$, $i--$}
 	{
  	$t=t*t\mo n$\\
 	\tcp{$i$th bit of $d$ is $1$}
  		\If{$d_i=1$}{
  		$t = a*t\mo n$
  		}
  	}
  	\Return t
\caption{Left-to-right square and multiply algorithm for computing modular exponentiation.}
\end{algorithm}
\end{frame}

\begin{frame}{Left-to-right square and multiply algorithm}
\begin{columns}[T] % align columns
\begin{column}{.4\textwidth}
{
\setlength{\interspacetitleruled}{0pt}%
\setlength{\algotitleheightrule}{0pt}%
\begin{algorithm}[H]
$t = 1$\\
	\For{$i=\ell_d-1$, $i\geq0$, $i--$}
 	{
  	$t=t*t\mo n$\\
 	\tcp{$i$th bit of $d$ is $1$}
  		\If{$d_i=1$}{
  		$t = a*t\mo n$
  		}
  	}
  	\Return t
\end{algorithm}
}
\end{column}%
\hfill%
\begin{column}{.6\textwidth}
\begin{example}
    Let $n=15,\ d=3=11_2,\  a=2$.
    Then
    \[
    a^d\mo n=2^3\mo15=8\mo15=8
    \]
 \begin{center}
        \begin{tabular}{c|c|c}
     $i$ & $d_i$  &  $t$  \\\hline
     $1$ & $1$    &  $2$  \\
     $0$ & $1$    &  $8$  
    \end{tabular}
    \end{center}
\end{example}
\end{column}%
\end{columns}
\end{frame}

\section{Implementations of Modular Multiplication}
\begin{frame}{\VideoName}
    \tableofcontents[currentsection]
\end{frame}

\begin{frame}{Modular operations}
    \begin{itemize}
        \item To have more efficient modular exponentiation implementations, we need to compute modular addition, subtraction, inverse, and multiplications.
        \item For modular addition and subtraction, we can just compute the corresponding integer operation and then perform a single reduction modulo the modulus.
        \item For inverse modulo an integer, we can utilize the extended Euclidean algorithm.
        \item We will look at one method for computing modular multiplication
    \end{itemize}
\end{frame}

\begin{frame}{Notations}
    \begin{itemize}
        \item $n$: an integer of bit length $\ell_n$, 
\begin{equation*}
    2^{\ell_n-1}\leq n<2^{\ell_n}.
\end{equation*}
        \item $a, b\in\ZZ_n$, in particular, $0\leq a,b<n$.
       \item $\omega$: the computer's word size
       \begin{itemize}
           \item for a 64-bit processor, the \textit{word size} is 64
       \end{itemize}
       \item Let $\kappa=\lceil \ell_n/\omega\rceil$, i.e. $(\kappa-1)\omega<\ell_n\leq \kappa\omega$.
      \item Then ($||$ indicates concatenation, $0\leq a_i< 2^{\omega}$)
\[
a=a_{\kappa-1}||a_{\kappa-2}||\dots||a_0,
\]
      \item Note that some $a_i$ might be $0$ if the bit length of $a$ is less than $\ell_n$. We have
\begin{equation*}
a=\sum_{i=0}^{\kappa-1}a_i(2^\omega)^i.
\end{equation*}
    \end{itemize}
\end{frame}

\begin{frame}{Notations}
     \begin{itemize}
        \item $n$: an integer of bit length $\ell_n$
        \item $a, b\in\ZZ_n$, in particular, $0\leq a,b<n$.
       \item $\omega$: the computer's word size
       \item Let $\kappa=\lceil \ell_n/\omega\rceil$, i.e. $(\kappa-1)\omega<\ell_n\leq \kappa\omega$.
      \item Then ($||$ indicates concatenation, $0\leq a_i< 2^{\omega}$)
\[
a=a_{\kappa-1}||a_{\kappa-2}||\dots||a_0,
\]
      \item Note that some $a_i$ might be $0$ if the bit length of $a$ is less than $\ell_n$. We have
\begin{equation*}
a=\sum_{i=0}^{\kappa-1}a_i(2^\omega)^i.
\end{equation*}
    \end{itemize}
\begin{example}
    $\omega=2$,  $a=13=1101_2$, $n=15$.
    Then
    \[
    \ell_n= 4,\quad
    \kappa=\lceil \ell_n/\omega\rceil=\lceil 4/2\rceil=2.
    \]
\end{example}
\end{frame}

\begin{frame}{Notations -- Example}
     \begin{itemize}
        \item $n$: an integer of bit length $\ell_n$ 
        \item $a, b\in\ZZ_n$, in particular, $0\leq a,b<n$.
       \item Let $\kappa=\lceil \ell_n/\omega\rceil$
      \item Then ($||$ indicates concatenation, $0\leq a_i< 2^{\omega}$)
\[
a=a_{\kappa-1}||a_{\kappa-2}||\dots||a_0,
\]
      \item Note that some $a_i$ might be $0$ if the bit length of $a$ is less than $\ell_n$. We have
\begin{equation*}
a=\sum_{i=0}^{\kappa-1}a_i(2^\omega)^i.
\end{equation*}
    \end{itemize}
\begin{example}
    $\omega=2$,  $a=13=1101_2$, $n=15$.
    Then $\ell_n=4$, $\kappa=2$.
    \[
    a_0=01_2=1,\quad a_1=11_2=3,\quad a=a_0(2^\omega)^0+a_1(2^{\omega})^1=1+3\times4=13
    \]
\end{example}
\end{frame}


\begin{frame}{Notations -- Example}
     \begin{itemize}
        \item $n$: an integer of bit length $\ell_n$, i.e. 
        \item $a, b\in\ZZ_n$, in particular, $0\leq a,b<n$.
       \item Let $\kappa=\lceil \ell_n/\omega\rceil$
      \item Then ($||$ indicates concatenation, $0\leq a_i< 2^{\omega}$)
\[
a=a_{\kappa-1}||a_{\kappa-2}||\dots||a_0,
\]
      \item Note that some $a_i$ might be $0$ if the bit length of $a$ is less than $\ell_n$. We have
\begin{equation*}
a=\sum_{i=0}^{\kappa-1}a_i(2^\omega)^i.
\end{equation*}
    \end{itemize}
\begin{example}
   $a=55=110111_2$, $n=69$, $\omega=2$.
$
  \ell_n=7,\ \kappa=\lceil \ell_n/\omega\rceil=\lceil7/2\rceil=4.
$
  \[
    a_0=11=3,\quad a_1=01=1,\quad a_2=11=3,\quad a_3=0.
    \]
    \[
    a=3\times(2^2)^0+1\times(2^2)^1+3\times(2^2)^2+0\times(2^2)^3=3+4+48+0=55.
    \]
\end{example}
\end{frame}


\begin{frame}{Blakley's method}
\begin{itemize}
\item We would like to compute
\[
R:=ab\mo n,\quad a,b\in\ZZ_n.
\]
\item We have discussed that
\begin{equation*}
a=\sum_{i=0}^{\kappa-1}a_i(2^\omega)^i,
\end{equation*}
where $0\leq a_i< 2^{\omega}$.
 \item The product $ab$ can be computed as follows
\end{itemize}
\[
t=ab=\left(\sum_{i=0}^{\kappa-1} a_i(2^\omega)^i\right)b=\sum_{i=0}^{\kappa-1}(2^\omega)^ia_ib,
\]
\end{frame}

\begin{frame}{}
    \begin{algorithm}[H]
\KwIn{$n,\ a,\ b$\tcp{$n\in\ZZ, n\geq2$ has bit length $\ell_n$; $a,b\in\ZZ_n$}}
\KwOut{$R=ab\mo n$}
$R=0$\\
\tcp{$\kappa=\lceil \ell_n/\omega\rceil$, where $\omega$ is the word size of the computer}
	\For{$i=\kappa-1$, $i\geq0$, $i--$}
 	{
 	$R=2^{\omega}R+a_ib$\\
        $R=R\mo n$
  	}
  	\Return $R$
	\caption{Blakely's method for computing modular multiplication.}
\end{algorithm}
\end{frame}

\begin{frame}{Blakely's method}
\begin{columns}[T] % align columns
\begin{column}{.4\textwidth}
{
\setlength{\interspacetitleruled}{0pt}%
\setlength{\algotitleheightrule}{0pt}%
\begin{algorithm}[H]
	\KwIn{$n,\ a,\ b$}
\KwOut{$R=ab\mo n$}
$R=0$\\
	\For{$i=\kappa-1$, $i\geq0$, $i--$}
 	{
 	$R=2^{\omega}R+a_ib$\\
        $R=R\mo n$
  	}
  	\Return $R$
\end{algorithm}
}
\end{column}%
\hfill%
\begin{column}{.6\textwidth}
Line~3,
\[
R\leq 2^{\omega}(n-1)+(2^{\omega}-1)(n-1)=(2^{\omega+1}-1)n-(2^{\omega+1}-1)<(2^{\omega+1}-1)n.
\]
Line~4 can be replaced by comparing $R$ with $n$ for $2^{\omega+1}-2$ times and subtract $n$ from $R$ in case $R\geq n$:
{
\setlength{\interspacetitleruled}{0pt}%
\setlength{\algotitleheightrule}{0pt}%
\begin{algorithm}[H]
\DontPrintSemicolon
	\For{$j=0,1,2\dots,2^{\omega+1}-2$}{
 \lIf{$R\geq n$}{
 $R=R-n$
 }
 \lElse{break}
    }
\end{algorithm}
}
in this way, we can avoid dividing by $n$
\end{column}%
\end{columns}
\end{frame}

\begin{frame}{Blakely's method -- Example}
\begin{columns}[T] % align columns
\begin{column}{.4\textwidth}
{
\setlength{\interspacetitleruled}{0pt}%
\setlength{\algotitleheightrule}{0pt}%
\begin{algorithm}[H]
	\KwIn{$n,\ a,\ b$}
\KwOut{$R=ab\mo n$}
$R=0$\\
	\For{$i=\kappa-1$, $i\geq0$, $i--$}
 	{
 	$R=2^{\omega}R+a_ib$\\
        $R=R\mo n$
  	}
  	\Return $R$
\end{algorithm}
}
\end{column}%
\hfill%
\begin{column}{.6\textwidth}
\begin{example}
    $\omega=2$,  $a=13=1101_2$, $b=5$, $n=15$ ($\ell_n=4$), $\kappa=2$.
    \[
    a_0=01_2=1,\quad a_1=11_2=3.
    \]
    For $i=1$,
    \[
    R=0+3\times5\mo 15=0\mo 15.
    \]
    For $i=0$, 
    \[
    R=0+1\times5\mo15=5\mo15
    \]
    We have the final result $13\times5\mo15=5$.
\end{example}
\end{column}%
\end{columns}
\end{frame}


\begin{frame}{Blakely's method -- Example}
\begin{columns}[T] % align columns
\begin{column}{.4\textwidth}
{
\setlength{\interspacetitleruled}{0pt}%
\setlength{\algotitleheightrule}{0pt}%
\begin{algorithm}[H]
	\KwIn{$n,\ a,\ b$}
\KwOut{$R=ab\mo n$}
$R=0$\\
	\For{$i=\kappa-1$, $i\geq0$, $i--$}
 	{
 	$R=2^{\omega}R+a_ib$\\
        $R=R\mo n$
  	}
  	\Return $R$
\end{algorithm}
}
\end{column}%
\hfill%
\begin{column}{.6\textwidth}
\begin{example}
   $a=55=110111_2$, $b=46$, $n=69$, $\omega=2$, $\ell_n=7$, $\kappa=4$, $
    a_0=11=3$, $a_1=01=1$, $a_2=11=3$, $a_3=0$
    \begin{eqnarray*}
        i=3\quad & \text{ line~3,} & R=0,\\
         & \text{ line~4,} & R=0,\\
        i=2\quad & \text{ line~3,} & R=3\times46=138,\\
         & \text{ line~4,} & R=138\mo69=0,\\
        i=1 \quad& \text{ line~3,} & R=1\times46=46,\\
         & \text{ line~4,} & R=46\mo69=46,\\
         i=0 \quad& \text{ line~3,} & R=2^2\times46+3\times46=322,\\
         & \text{ line~4,} & R=322\mo69=46.
    \end{eqnarray*}
\end{example}
\end{column}%
\end{columns}
\end{frame}

\begin{frame}{Final remarks}
    \begin{itemize}
        \item Currently, a few hundred qubits (a quantum counterpart to the classical bit) are possible for a quantum computer
        \item To break RSA, thousands of qubits are required
        \item Post-quantum public key cryptosystems are being proposed to protect communications after a sufficiently strong quantum computer is built
        \item Various public key cryptosystems based on different problems
        \item Various digital signature designs
    \end{itemize}
\end{frame}

\begin{frame}{Section summary}
    \begin{itemize}
        \item Classical ciphers, e.g. shift cipher
        \item Modern ciphers: DES, AES, PRESENT
        \item Implementation of symmetric block ciphers - T-tables for AES, combining sBoxLayer and pLayer for PRESENT
        \item RSA, RSA signatures
        \item Implementation of RSA - square and multiply algorithm, Blakely's method
    \end{itemize}
\end{frame}