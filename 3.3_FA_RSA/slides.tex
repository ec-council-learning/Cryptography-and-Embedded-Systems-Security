\begin{frame}{Remarks}
    \begin{itemize}
        \item Unlike fault attacks on symmetric block ciphers, attacks on public key ciphers depend on the underlying intractable problem and we do not have a systematic methodology.
        \item However, the general attack concept can be applied to ciphers based on similar intractable problems.
        \item We will discuss an attack on RSA signatures in this video and the corresponding countermeasure in the next video
        \item The attack we will see can also be applied to RSA decryption process
    \end{itemize}
\end{frame}

\section{RSA Signatures and Implementations}
\begin{frame}{\VideoName}
    \tableofcontents[currentsection]
\end{frame}

\begin{frame}{RSA}
\begin{definition}[RSA]
Let $n=pq$, where $p,q$ are distinct prime numbers.
Let $\mathcal{P}=\mathcal{C}=\ZZ_n$, $\mathcal{K}=\ZZ_{\varphi(n)}^*-\Set{1}$.
For any $e\in \mathcal{K}$, define encryption
\[
    E_e:\ZZ_n\to\ZZ_n,\quad m\mapsto m^e\mo n,
\]
and the corresponding decryption
\[
    D_d:\ZZ_n\to\ZZ_n,\quad c\mapsto c^d\mo n,
\]
where $de\mo\varphi(n)=1$.
The cryptosystem $(\mathcal{P},\mathcal{C},\mathcal{K},\mathcal{E},\mathcal{D})$, where $\mathcal{E}=\Set{E_e:e\in\mathcal{K}}$, $\mathcal{D}=\Set{D_d:d\in\mathcal{K}}$, is called \textit{RSA}.
\end{definition}
\begin{itemize}
    \item $\varphi(n)=(p-1)(q-1)$
    \item Public key: $n,e$, RSA modulus, encryption exponent
    \item Private key: $d$, decryption exponent
\end{itemize}
\end{frame}

\begin{frame}{RSA signatures}
    \begin{itemize}
        \item To use RSA for digital signature, we again let $p$ and $q$ be two distinct primes.
        \item $n=pq$, choose $e\in\ZZ_{\varphi(n)}^*$, compute $d$ such that $de\mo\varphi(n)=1$
        \item The public key consists of $e$ and $n$.
        \item $d$ is the private key.
        \item $p$, $q$ and $\varphi(n)$ should be kept secret.
    \end{itemize}
\end{frame}

\begin{frame}{RSA signatures}
To sign a message $m$, Alice computes the signature
\[
s=m^d\mo n.
\]
Then Alice sends both $m$ and $s$ to Bob.
To verify the signature, Bob computes
\[
s^e\mo n.
\]
If $s\equiv m\mo n$, then the verification algorithm outputs true, and false otherwise.
\begin{itemize}
    \item Up to now, the only method known to compute $s$ from $m\mo n$ is using $d$, so if the verification algorithm outputs true, Bob can conclude that Alice is the owner of $d$.
\end{itemize}
\end{frame}

\begin{frame}{Notations}
    \begin{itemize}
        \item $n$ has bit length $\ell_n$, 
\begin{equation*}
    2^{\ell_n-1}\leq n<2^{\ell_n}.
\end{equation*}
        \item $a, b\in\ZZ_n$, in particular, $0\leq a,b<n$.
       \item $\omega$: the computer's word size
       \begin{itemize}
           \item for a 64-bit processor, the \textit{word size} is 64
       \end{itemize}
       \item Let $\kappa=\lceil \ell_n/\omega\rceil$, i.e. $(\kappa-1)\omega<\ell_n\leq \kappa\omega$.
      \item Then ($||$ indicates concatenation, $0\leq a_i< 2^{\omega}$)
\[
a=a_{\kappa-1}||a_{\kappa-2}||\dots||a_0,
\]
      \item Note that some $a_i$ might be $0$ if the bit length of $a$ is less than $\ell_n$. We have
\begin{equation*}
a=\sum_{i=0}^{\kappa-1}a_i(2^\omega)^i.
\end{equation*}
    \end{itemize}
\end{frame}

\begin{frame}{Blakley's method}
\begin{itemize}
\item We would like to compute
\[
R=ab\mo n.
\]
\item Since
\begin{equation*}
a=\sum_{i=0}^{\kappa-1}a_i(2^\omega)^i,
\end{equation*}
where $0\leq a_i< 2^{\omega}$.
 \item The product $ab$ can be computed as follows
\end{itemize}
\[
t=ab=\left(\sum_{i=0}^{\kappa-1} a_i(2^\omega)^i\right)b=\sum_{i=0}^{\kappa-1}(2^\omega)^ia_ib,
\]
\end{frame}

\begin{frame}{}
    \begin{algorithm}[H]
\KwIn{$n,\ a,\ b$\tcp{$n\in\ZZ, n\geq2$ has bit length $\ell_n$; $a,b\in\ZZ_n$}}
\KwOut{$ab\mo n$}
$R=0$\\
\tcp{$\kappa=\lceil \ell_n/\omega\rceil$, where $\omega$ is the word size of the computer}
	\For{$i=\kappa-1$, $i\geq0$, $i--$}
 	{
 	$R=2^{\omega}R+a_ib$\\
        $R=R\mo n$
  	}
  	\Return $R$
	\caption{Blakely's method for computing modular multiplication.}
\end{algorithm}
\end{frame}

\begin{frame}{Right-to-left Square and Mulitply Algorithm}
    \begin{algorithm}[H]
\KwIn{$n,\ m,\ d$}
\KwOut{$s=m^d\mo n$}
$s= 1$, $t = m$\\
	\For{$i=0$, $i<\ell_d$, $i++$}
 	{
 	\tcp{$i$th bit of $d$ is $1$}
  		\If{$d_i=1$}{
            \tcp{multiply by $m^{2^i}$}
  		$s=s*t\mo n$\\
  		}
  		\tcp{$t=m^{2^{i+1}}$}
  		$t=t*t\mo n$\\
  	}
  	\Return $s$\\
	\caption{Computing RSA signature with the right-to-left square and multiply algorithm.}
\end{algorithm}
\end{frame}

\begin{frame}{Right-to-left square and multiply algorithm with Blakely's method}
    \begin{itemize}
        \item Since $\ell_n$ is the bit length of $n$, the bit lengths of the variables $s$ and $t$ are at most $\ell_n$.
        \item $\omega$ is the computer's word size
        \[
        \kappa=\lceil \ell_n/\omega\rceil
        \]
        \item We can write
\[
s=\sum_{j=0}^{\kappa-1}s_j(2^{\omega})^j,\quad t=\sum_{j=0}^{\kappa-1}t_j(2^{\omega})^j.
\]
    \end{itemize}
\end{frame}

\begin{frame}{Right-to-left square and multiply algorithm with Blakely's method}
Input: $n,m,d$. $\quad$ Output: $m^d\mo n$
{\small
{
\setlength{\interspacetitleruled}{0pt}%
\setlength{\algotitleheightrule}{0pt}%
    \begin{algorithm}[H]
% \KwIn{$n,\ m,\ d$
% }
% \KwOut{$m^d\mo n$}
$s= 1,\quad$
$t = m$\\
	\For{$i=0$, $i<\ell_d$, $i++$}
 	{
  		\If{$d_i=1$}{
    \tcp{lines~\ref{line:alg:fablakelyrtlsam:R=0rt} --~\ref{line:alg:fablakelyrtlsam:r=R} implement $s=s*t\mo n$}
            $R=0$\label{line:alg:fablakelyrtlsam:R=0rt}\\
            \For{$j=\kappa-1$, $j\geq0$, $j--$}{
            $R=2^\omega R+s_jt$\label{line:alg:fablakelyrtlsam:sjt}\\
            $R=R\mo n$
            }
            $s=R$\label{line:alg:fablakelyrtlsam:r=R}
  		}
            $R=0$\label{line:alg:fablakelyrtlsam:R=0tt}\tcp{lines~\ref{line:alg:fablakelyrtlsam:R=0tt} --~\ref{line:alg:fablakelyrtlsam:t=R} implement $t=t*t\mo n$}
            \For{$j=\kappa-1$, $j\geq0$, $j--$}{
            \label{line:alg:fablakelyrtlsam:for2}
            $R=2^\omega R+t_jt$\\
            $R=R\mo n$
            }
            $t=R$\label{line:alg:fablakelyrtlsam:t=R}\\
  	}
  	\Return $s$\\
\end{algorithm}}}
\end{frame}

\begin{frame}{Right-to-left square and multiply algorithm with Blakely's method}
\begin{columns}[T] % align columns
\begin{column}{.45\textwidth}
{\small
{
\setlength{\interspacetitleruled}{0pt}%
\setlength{\algotitleheightrule}{0pt}%
    \begin{algorithm}[H]
$s= 1,\quad$
$t = m$\\
	\For{$i=0$, $i<\ell_d$, $i++$}
 	{
  		\If{$d_i=1$}{
            $R=0$\\
            \For{$j=\kappa-1$, $j\geq0$, $j--$}{
            \label{line:alg:fablakelyrtlsam:for1}
            $R=2^\omega R+s_jt$\\
            $R=R\mo n$
            }
            $s=R$
  		}
            $R=0$\\
            \For{$j=\kappa-1$, $j\geq0$, $j--$}{
            $R=2^\omega R+t_jt$\\
            $R=R\mo n$
            }
            $t=R$\\
  	}
  	\Return $s$\\
\end{algorithm}}}
\end{column}%
\hfill%
\begin{column}{.58\textwidth}
\begin{example}
\[
n=15,\quad d=3=11_2,\quad m=2,\quad \ell_d=2,\quad \kappa=2.
\]
\[
\begin{array}{llll}
i=0,d_0=1 & \text{line 5}  & j=1 & R=s_1t\mo n=0\\
   &  & j=0 & R=s_0t\mo n=2\\
   & \text{line 8}  & s=2 & s_0=10, s_1=00\\
  &  \text{line 10} & j=1 & R=0\\
   & & j=0 & R=4\\
   &\text{line 13} & t=4 & t_{0}=00, t_{1}=01\\
i=1,d_1=1 & \text{line 5} & j=1 & R=0\\
& & j=0 & R=8\\
& \text{line 9} & s=8
\end{array}
\]
\[
m^d\mo n=2^3\mo15=8
\]
\end{example}
\end{column}%
\end{columns}
\end{frame}


\section{Safe Error Attack}
\begin{frame}{\VideoName}
    \tableofcontents[currentsection]
\end{frame}

\begin{frame}{The attack}
    \begin{itemize}
        \item Targets square and multiply algorithm
        \item Assumes Blakely's method used for modular multiplication implementation
        \item The attack exploits the knowledge of whether an intermediate faulty value is used or not by observing whether the final output is changed, thus the name \textit{safe error attack}~\footnote{Yen, S. M., \& Joye, M. (2000). Checking before output may not be enough against fault-based cryptanalysis. IEEE Transactions on computers}.
       \item Since only knowing whether the output is changed or not is enough, if a countermeasure that repeats the computation, compares the final results, and outputs an error when a fault is detected is implemented, the safe error attack still applies.
    \end{itemize}
\end{frame}

\begin{frame}{Recall -- Notations}
    \begin{itemize}
        \item $n$ has bit length $\ell_n$, 
\begin{equation*}
    2^{\ell_n-1}\leq n<2^{\ell_n}.
\end{equation*}
        \item $a, b\in\ZZ_n$, in particular, $0\leq a,b<n$.
       \item $\omega$: the computer's word size
       \begin{itemize}
           \item for a 64-bit processor, the \textit{word size} is 64
       \end{itemize}
       \item Let $\kappa=\lceil \ell_n/\omega\rceil$, i.e. $(\kappa-1)\omega<\ell_n\leq \kappa\omega$.
      \item Then ($||$ indicates concatenation, $0\leq a_i< 2^{\omega}$)
\[
a=a_{\kappa-1}||a_{\kappa-2}||\dots||a_0,
\]
      \item Note that some $a_i$ might be $0$ if the bit length of $a$ is less than $\ell_n$. We have
\begin{equation*}
a=\sum_{i=0}^{\kappa-1}a_i(2^\omega)^i.
\end{equation*}
    \end{itemize}
\end{frame}

\begin{frame}{Attack on a simple algorithm}
    \begin{algorithm}[H]
\KwIn{$n,\ a,\ b,\ c$\tcp{$a,b\in\ZZ_n$; $c=0,1$}}
\KwOut{$ab\mo n$ if $c=1$ and $a$ otherwise}
\If{$c=1$}{
$R=0$\\
\tcp{$\kappa=\lceil \ell_n/\omega\rceil$, where $\omega$ is the computer's word size}
	\For{$i=\kappa-1$, $i>=0$, $i--$}
 	{
 	$R=2^\omega R+a_ib$\\
 	$R=R\mo n$\\
  	}
  	$a=R$\\
  }
  	\Return $a$
\caption{An algorithm involving computing modular multiplication with Blakely's method.}
\end{algorithm}
\end{frame}

\begin{frame}{Attack on a simple algorithm}
\begin{columns}[T] % align columns
\begin{column}{.4\textwidth}
{
\setlength{\interspacetitleruled}{0pt}%
\setlength{\algotitleheightrule}{0pt}%
\begin{algorithm}[H]
\KwIn{$n,\ a,\ b,\ c$}
\KwOut{$ab\mo n$ if $c=1$ and $a$ otherwise}
\If{$c=1$}{
$R=0$\\
	\For{$i=\kappa-1$, $i>=0$, $i--$
 \label{line:alg:farsaexblakely:loop}
 }
 	{
 	$R=2^\omega R+a_ib$\\
 	$R=R\mo n$\\
  	}
  	$a=R$\\
  }
  	\Return $a$
\end{algorithm}
}
\end{column}%
\hfill%
\begin{column}{.6\textwidth}
 \begin{itemize}
        \item Attacker has the knowledge of the correct output for a pair of $a$ and $b$.
       \item Can rerun the algorithm with the same input, inject fault, and observe the output.
        \item Suppose $c=1$ and a fault is injected during the loop starting from line~\ref{line:alg:farsaexblakely:loop} in the register containing $a_{i_0}$ when $i<i_0$ -- the fault in $a_{i_0}$ will not affect the output since $a_{i_0}$ is used when $i$ is equal to $i_0$.
       \item If $c=0$ and a fault is injected in the register containing $a_{i_0}$ during the computation, then the final result will be faulty since the faulty value in $a$ will be returned.
    \end{itemize}
\end{column}%
\end{columns}
\end{frame}


\begin{frame}{Attack on a simple algorithm}
    \begin{columns}[T] % align columns
\begin{column}{.42\textwidth}
{
\setlength{\interspacetitleruled}{0pt}%
\setlength{\algotitleheightrule}{0pt}%
\begin{algorithm}[H]
\KwIn{$n,\ a,\ b,\ c$}
\KwOut{$ab\mo n$ if $c=1$ and $a$ otherwise}
\If{$c=1$}{
$R=0$\\
	\For{$i=\kappa-1$, $i>=0$, $i--$}
 	{
 	$R=2^\omega R+a_ib$\\
 	$R=R\mo n$\\
  	}
  	$a=R$\\
  }
  	\Return $a$
\end{algorithm}
}
\end{column}%
\hfill%
\begin{column}{.63\textwidth}
 \begin{itemize}
        \item Attacker knows: correct output, $a$, $b$.
        \item If $c=1$ and a fault is injected during the loop starting from line~\ref{line:alg:farsaexblakely:loop} in the register containing $a_{i_0}$ when $i<i_0$ -- output will not be affected
       \item If $c=0$, output will be faulty
       \item Now, if the attacker does not know the value of $c$ and would like to recover it by fault injection attacks.
       \item Assume that $c=1$ and the loop in line~\ref{line:alg:farsaexblakely:loop} is executed.
        \item Injects fault in $a_{i_0}$ at the time when $i$ is less than $i_0$.
       \item Compares the output with the correct one and recovers the value of $c$ -- if the output is correct, $c=1$; otherwise $c=0$.
    \end{itemize}
\end{column}%
\end{columns}
\end{frame}

\begin{frame}{Safe error attack on RSA Signatures}
\begin{columns}[T] % align columns
\begin{column}{.45\textwidth}
{\small
{
\setlength{\interspacetitleruled}{0pt}%
\setlength{\algotitleheightrule}{0pt}%
    \begin{algorithm}[H]
$s= 1,\quad$
$t = m$\\
	\For{$i=0$, $i<\ell_d$, $i++$}
 	{
  		\If{$d_i=1$}{
            $R=0$\\
            \For{$j=\kappa-1$, $j\geq0$, $j--$}{
            $R=2^\omega R+s_jt$\\
            $R=R\mo n$
            }
            $s=R$
  		}
            $R=0$\\
            \For{$j=\kappa-1$, $j\geq0$, $j--$}{
            $R=2^\omega R+t_jt$\\
            $R=R\mo n$
            }
            $t=R$\\
  	}
  	\Return $s$\\
\end{algorithm}}}
\end{column}%
\hfill%
\begin{column}{.58\textwidth}
\begin{itemize}
        \item Suppose $d_i=1$ and a fault is injected during the $i$th iteration of the outer loop and at the time when $j<j_0$ during the loop starting from line~\ref{line:alg:fablakelyrtlsam:for1}, in the register containing $s_{j_0}$ -- correct output 
       \item If $d_i=0$ and a fault is injected during the $i$th iteration of the outer loop in the register containing $s_{j_0}$ -- faulty output
       \item Similarly to the attack on the simple algorithm, the attacker first assumes $d_i=1$, and injects fault in $s_{j_0}$ at the time corresponding to $j<j_0$.
\item If the final result is not changed, the attacker can conclude that $d_i=1$, otherwise, $d_i=0$.
    \end{itemize}
\end{column}%
\end{columns}
\end{frame}

\begin{frame}{Safe error attack on RSA signatures -- Example}
\begin{columns}[T] % align columns
\begin{column}{.45\textwidth}
{\small
{
\setlength{\interspacetitleruled}{0pt}%
\setlength{\algotitleheightrule}{0pt}%
    \begin{algorithm}[H]
$s= 1,\quad$
$t = m$\\
	\For{$i=0$, $i<\ell_d$, $i++$}
 	{
  		\If{$d_i=1$}{
            $R=0$\\
            \For{$j=\kappa-1$, $j\geq0$, $j--$}{
            \label{line:alg:fablakelyrtlsam:for1}
            $R=2^\omega R+s_jt$\\
            $R=R\mo n$
            }
            $s=R$
  		}
            $R=0$\\
            \For{$j=\kappa-1$, $j\geq0$, $j--$}{
            $R=2^\omega R+t_jt$\\
            $R=R\mo n$
            }
            $t=R$\\
  	}
  	\Return $s$\\
\end{algorithm}}}
\end{column}%
\hfill%
\begin{column}{.58\textwidth}
\begin{example}
\[
n=15,\quad d=3=11_2,\quad m=2,\quad \ell_d=2,\quad \kappa=2.
\]
\[
\begin{array}{llll}
i=0,d_0=1 & \text{line 5}  & j=1 & R=\textcolor{blue}{s_1}t\mo n=0\\
   &  & j=0 & R=s_0t\mo n=2\\
   & \text{line 8}  & s=2 & s_0=10, \textcolor{orange}{s_1}=00\\
  &  \text{line 10} & j=1 & R=0\\
   & & j=0 & R=4\\
   &\text{line 13} & t=4 & t_{0}=00, t_{1}=01\\
i=1,d_1=1 & \text{line 5} & j=1 & R=0\\
& & j=0 & R=8\\
& \text{line 9} & s=8
\end{array}
\]
Makes guess $d_0=1$, injects fault into $s_1$ when $i=0$, $j=0$ (line 5)
\end{example}
\end{column}%
\end{columns}
\end{frame}

\begin{frame}{Safe error attack on RSA signatures -- Example}
    \begin{example}
\[
n=15,\quad d=3=11_2,\quad m=2,\quad \ell_d=2,\quad \kappa=2.
\]
\[
\begin{array}{llll}
i=0,d_0=1 & \text{line 5}  & j=1 & R=\textcolor{blue}{s_1}t\mo n=0\\
   &  & j=0 & R=s_0t\mo n=2\\
   & \text{line 8}  & s=2 & s_0=10, \textcolor{orange}{s_1}=00\\
  &  \text{line 10} & j=1 & R=0\\
   & & j=0 & R=4\\
   &\text{line 13} & t=4 & t_{0}=00, t_{1}=01\\
i=1,d_1=1 & \text{line 5} & j=1 & R=0\\
& & j=0 & R=8\\
& \text{line 9} & s=8
\end{array}
\]
\begin{itemize}
\item Makes guess $d_0=1$, injects fault into $s_1$ when $i=0$, $j=0$ (line 5)
    \item $s_1$ is used (blue \textcolor{blue}{$s_1$}) before $j=0$ and reassigned value in line~8 (orange \textcolor{orange}{$s_1$}).
\item Thus the computations are not affected and the final result is unchanged.
\item The attacker can conclude that $d_0=1$.
\end{itemize}

\end{example}
\end{frame}

\begin{frame}{Safe error attack on RSA signatures}
    \begin{itemize}
\item The attacker can repeat the attack for different values of $i$ to recover the entire private key.
\item Similar techniques can also be applied to attack the left-to-right square and multiply algorithm with Blakely's method\footnote{Yen, S. M., \& Joye, M. (2000). Checking before output may not be enough against fault-based cryptanalysis. IEEE Transactions on Computers}.
    \end{itemize}
\end{frame}


\begin{frame}{Final Remarks}
    \begin{itemize}
        \item The very first fault attack, the Bellcore attack, on cryptographic implementations was proposed by Boneh et al.\footnote{Boneh, D., DeMillo, R. A., \& Lipton, R. J. (1997, May). On the importance of checking cryptographic protocols for faults. In International conference on the theory and applications of cryptographic techniques (pp. 37-51). Springer, Berlin, Heidelberg.} for attacking RSA signatures.
        \begin{itemize}
            \item Given one faulty signature, attacker can factor the RSA modulus
            \item In this paper, the authors also discussed an attack aiming at the intermediate values of the square and multiply algorithm to recover the private key.
        \end{itemize}
        \item Even though no systematic methodologies exist for fault attacks on public key ciphers, the general attack concept can be applied to a different cipher based on a similar intractable problem.
        \begin{itemize}
            \item For example, an attack on the square and multiply algorithm can be applied to attack discrete logarithm-based ciphers\footnote{Bao, Feng, et al. "Breaking public key cryptosystems on tamper resistant devices in the presence of transient faults." Security Protocols: 5th International Workshop Paris, France, April 7–9, 1997 Proceedings 5. Springer Berlin Heidelberg, 1998.}.
        \end{itemize}
    \end{itemize}
\end{frame}