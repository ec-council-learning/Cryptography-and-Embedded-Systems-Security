\begin{frame}{Introduction}
    \begin{itemize}
        \item Cryptography is an indispensable tool used to protect the information in computing systems
        \item Cryptographic algorithms in use nowadays are considered secure in theory.
        \item But in the real world, these algorithms are implemented on physical devices in the form of integrated circuits.
        \item These circuits have their physical properties, such as power consumption dependent on the processed data, emanation of electromagnetic waves, and susceptibility to computational errors due to environmental influences.
        \item To evaluate the security level of cryptographic implementations, it is necessary to include the physical security assessment.
    \end{itemize}
\end{frame}

\begin{frame}{Cryptography}
    \begin{itemize}
        \item Classical ciphers, e.g. shift cipher
        \item Modern ciphers: DES, AES, PRESENT
        \item Implementation of symmetric block ciphers - T-tables for AES, combining sBoxLayer and pLayer for PRESENT
        \item RSA, RSA signatures
        \item Implementation of RSA - square and multiply algorithm, Blakely's method
    \end{itemize}
\end{frame}

\begin{frame}{Fault attacks and countermeasures}
    \begin{itemize}
        \item Fault attacks on symmetric block ciphers - differential fault analysis, diagonal fault attack on AES
        \item Countermeasures against fault attacks on symmetric block ciphers - encoding-based countermeasure
        \item Fault attacks on RSA and RSA signatures - safe error attack
        \item Countermeasures against fault attacks on RSA and RSA signatures - a simple change of multiplicand and multiplier against safe error attack
        \item Fault injection techniques vary greatly in their effectiveness and cost
    \end{itemize}
\end{frame}

\begin{frame}{Experimental Aspects of Side-Chanel Analysis}
    \begin{itemize}
        \item Power consumption side-channel leakages come from the switching power in CMOS-based circuits
        \item Power analysis attacks utilizes osciloscope for capturing the traces
        \item The captured power consumption, or so-called leakage, is dependent on both operation and data in the DUT
        \item Leakage assessment can be used to assess the a cryptographic implementation’s security to determine if it leaks information through side channels.
        \item SNR computations can help to identify the point-of-interest for our attack
    \end{itemize}
\end{frame}

\begin{frame}{Simple and Differential Power Analysis}
    \begin{itemize}
        \item SPA on RSA - targets square and multiply algorithm
        \item Countermeasure against SPA on RSA - square and multiply-always algorithm
        \item DPA on symmetric block ciphers - attack on PRESENT
        \item Profiled DPA on symmetric block ciphers - first identify POI
        \item Countermeasures against DPA on symmetric block ciphers - masking for PRESENT
    \end{itemize}
\end{frame}

\begin{frame}{Other hardware attack techniques}
    \begin{itemize}
        \item Timing attacks exploit variations in execution time or other timing-related behaviors to infer sensitive information
        \item Cache attacks exploit timing variations and access patterns in a computer's cache memory to extract sensitive information
        \item Hardware Trojans are malicious modifications introduced into electronic circuits during design or manufacturing that can alter functionality or compromise security, often leading to vulnerabilities or failures in critical systems.
        \item Cold boot attacks exploit the remanence effect of RAM to recover unencrypted data from RAM. 
    \end{itemize}
\end{frame}